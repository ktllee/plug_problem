\documentclass[10pt]{article}
%\usepackage[textheight=10in, textwidth=7in]{geometry}

\usepackage{amsmath}
\usepackage{amsthm}
\usepackage{amsfonts} % to get \mathbb letters
\usepackage{xcolor}
\usepackage{array}
\usepackage{graphicx}
\usepackage{tabularx}% http://ctan.org/pkg/tabularx
\usepackage{booktabs}% http://ctan.org/pkg/booktabs
\usepackage{hyperref}
\usepackage{verbatim}
\usepackage{comment}
\usepackage{amssymb}
\usepackage{mathrsfs}

\newtheorem{theorem}{Theorem}
\newtheorem{definition}[theorem]{Definition}
\newtheorem{lemma}[theorem]{Lemma}
\newtheorem{conjecture}[theorem]{Conjecture}
\newtheorem{corollary}[theorem]{Corollary}
\newtheorem{remark}[theorem]{Remark}
\newtheorem{question}{Question}
\newtheorem{example}[theorem]{Example}
\numberwithin{equation}{section}

\newenvironment{anote}
               {{\textcolor{blue}{Note:}}
                 \itshape
               }
               {}
                 
\newcommand{\plug}[1]{%
\mbox{{\textbf\texttt #1}}
}

\newcommand{\puzzle}[1]{%
   \ensuremath{
   \mathbb{#1}
   }
}

\title{Plug and Play}
\author{
Ethan D. Bolker \\
Debra K. Borkovitz\\
  Katelyn Lee\\
  Adam Salachi
}
\date{\today}

\begin{document}

\maketitle

\section{Introduction}

\begin{anote}
Describe the source of the problem and the group makeup
\end{anote}

\begin{quote}
    Given a sequence $0,1,1,2,2,3,3,...,n,n$ find the number of ways to arrange the elements in the sequence so that there's exactly one digit between the two $1$'s, two digits between the two $2$'s, three digits between the two $3$'s, so on and so forth, and there are exactly $n$ digits between the two $n$'s.

For $n=2$, a possible solution would be $12102$. For $n=5$, a possible solution would be $53141352402$.\cite{originalproblem}
\end{quote}

\begin{anote}
Reformulate and generalize to plugs on a strip. Pictures of our 3d printed incarnations.
\end{anote}
\section{Plugs, Strips and Puzzles}

Our mathematical model for a plug is a bit string, for example
\plug{1011}. Our current convention is that there are no leading or
trailing \plug{0} bits. (We may want to relax that convention at some
time.)

The \emph{length} of a plug is the length of the string; this one has
length $4$. The \emph{number of prongs} is the number of \plug{1} bits
--- $3$ in this example.

Each plug has a \emph{plugnumber} when we
interpret its bit string as a binary integer. So \plug{1011} is
number $1 + 4 + 8 = 13$.%
\footnote{This convention reads increasing bit significance from left
  to right, so the first bit is the units bit. We can change our minds
  and set the plugnumber for \plug{1011} to $1+2+8 = 11$ if we wish,
  but let's decide soon and stick to our decision.}

Since plugs begin and end with \plug{1} bits,
the plugs of length $n$ have plugnumbers the odd integers between
$2^{n-1} +1$ and $2^n -1$.
\footnote{This equivalence will need revision if we allow leading and
  trailing \plug{0}s in plug bitstrings.}

We will want ways to talk about plugs other than specifying their bit
strings or plugnumbers. In any context we can give a plug any name we
like, or have variables whose values can be plugs.

The Stackexchange question that triggered our project asked about
the $2$ prong plugs. We'll call those the \emph{classic plugs} and give
them their own names: $T_k$ for the two prong plug of length $k$, It
has $k-2$ \plug{0} bits between its $2$ end prongs. $T_k$ has plugnumber $2^{k-1} + 1$.
 The stackexchange question uses the single prong plug \plug{1}{} instead of the mathematically more natural length $2$ two prong plug $T_2$ with no gap between the prongs.
 
 A \emph{power strip} (or just a \emph{strip}) models a place to plug
 in plugs. Think of it as a finite sequence of slots some of
 which are occupied by the prongs of plugs. So a power strip
is a pair consisting of an array of slots and a set of (plug, offset)
pairs such that all the plugs can be inserted simultaneously at
 the specified offsets. It is a \emph{solved} if
 every slot in the strip is filled.

A \emph{plug puzzle}, or, for us, just a puzzle, specifies the types amd numbers of plugs you are allowed to use to fill a strip --- something like
\begin{equation*}
    \puzzle{E}(\text{list of allowable plug types},\text{restrictions on numbers of plugs of each type})
\end{equation*}

A \emph{solution} of length $n$ is a strip with $n$ slots filled with plugs that are consistent with the restrictions in the puzzle specification.


Given a puzzle we try to understand the number and shape of solutions. often as
  a function of the size of the strip.

\begin{anote}
Remember to say somewhere (perhaps not here) that plugs and solutions have a left to right direction. You can't turn them around. Perhaps add an electric system not quite analogy.
\end{anote}

The stackexchange post posed a sequence of puzzles:
\begin{equation*}
\mathbb{T}_k = \puzzle{E}( \{\text{the one prong plug}, T_3, T_4, \ldots. T_k\}, \text{exactly 1 of each } ).
\end{equation*}
The only possible solutions are strips of length $2k+1$. The question asks for an efficient algorithm to count them.

 \section{Anything goes}
 
That puzzle spurred our investigations,but it is much too hard to start out with. In particular, we discovered that restricting the number of plugs of each type is a stumbling block. So for a while we will study just puzzles that specify the allowed plug types, with no limit on the number of each that can appear in a solution.

In the \emph{anything goes}  puzzle $\puzzle{A}$  you may use any plugs as oftem as you wish:
\begin{equation*}
  \puzzle{A}  = \puzzle{E}(\text{all plug types} )
\end{equation*}
Counting solutions begins to put us in touch with some classical notions
in combinatorics.

\begin{theorem} Length $n$ solutions for the plug
  puzzle that allows arbitrary many instances of any plug
correspond to the partitions of the   $n$-element set $[n] = \{1,2,
\ldots, n \}$. 
\end{theorem}

\begin{anote}
Example here --- wait until the good \LaTeX{} plug representation
  is done.
\end{anote}

The \href{https://en.wikipedia.org/wiki/Bell_number}{Bell numbers}
$\{1, 2, 3, 5, 15, 52, \ldots$  count these.

\section{Factoring}

There is more information in a a solved  strip than the shape of
the partition that defines it. For example, two copies of \plug{101}
solve a strip of length $4$. So do two copies of \plug{11}. Both solutions correspond to a partition of the four element set of slots
into two sets of two slots each, but their geometry is different.

In a  power strip of length $n$ there are $n-1$ \emph{gaps} between
slots. The \emph{thickness} or \emph{thickness array} is a the
sequence of integers that counts the number of plugs that cover each
of the gaps. A solution (that is, a filled strip) is \emph{prime} or \emph{atomic}if
its thickness is never $0$. Any solution is a concatenation of prime
filled strips: the \emph{factors}.

When a puzzle specifies no restrictions on the number of each kind of plug then factors and products (concatenations) of solutions are themselves solutions.  When that  happens we can relate the number of prime solutions and the number of solutions.

Let   $S(n)$ be the number of solved strips of length $n$ and
 $S'(n)$ the number of prime solutions of length $n$. 
 
Let $P(n)$  be the number of solved strips that use $n$ plugs and
  $P'(n)$ the number of prime solutions that use $n$ plugs.

NOT QUITE RIGHT YET.  $P$ and $P'$ must be carefully defined to count the number of prime plugs actually used in solutions of some length or bounded length. To see why, note that for the anything goes puzzle there are infinitely many one plug primes: the plugs with all bits \plug{1}.

When necessary we will write $S(\puzzle{E},n)$ when we need to make clear which puzzle we are counting.

Some of these counts may be $0$. Consider the
puzzle that uses just the classic  plug \plug{11}.
Then $S(n) = 1$ if $n$ is even, $0$ if $n$ is odd. $S'(2) =1$ and
$0$  otherwise.


We did a brute force calculation for the number of prime solutions to the anything goes puzzle, and found $\{1,1,2,6,22, \ldots\}$. That sequence seems to be
\href{https://oeis.org/A074664}{A074664} in
\href{https://oeis.org}{The On-Line Encyclopedia of Integer
  Sequences}. The comments there about permutations and partitions
strongly support

\begin{conjecture}
The prime solutions to problem $\puzzle{A}$ are characterized by the
equivalent conditions defining the OEIS sequence A074664.
\end{conjecture}

This easy theorem might help prove the conjecture. It shows that each of $S$ and $S'$ (or $P$ and $P'$) determines the other.

\begin{theorem}\label{thm:primerecursion}
Suppose that every factor of a solved strip solves its smaller strip and any concatenation of solved strips is a solution.

 Then the  first $k$ for which $S(k) > 0$ is the same as the first $k$ for which $S'(k) > 0$. For that $k$ we have $S(k) = S'(k)$ and for all $n$
 
\begin{equation}\label{eq:convolution}
   S(n) = S'(1)S(n-1) +  S'(2)S(n-2) + \cdots +   S'(n-1)S(1) + S'(n) .
\end{equation}

The same assertions  hold for the counts $P(n)$ and $P'(n)$.
\end{theorem}
\begin{proof}
Each solved strip begins with a prime solution of length $k \le n$ which can be
chosen $S'(k)$ ways and ends with one of the $S(n-k)$ solved strips of
length $n-k$. 
Replace ``length" by ``number of plugs" to prove the second assertion.
\end{proof}

Equation~\ref{eq:convolution} id as kind of convolution. It appears in the OEIS discussion of \href{https://oeis.org/A074664}{OEIS A074664}. In Section~\ref{sec:convolution} we discuss it further.

\begin{anote}
So far we've used thickness only to find factors. But there is
probably more information to exploit. What's the maximum thickness?
What patterns are possible in the thickness vector?
\end{anote}

\section {Disallowing the Plug $1$}

\begin{anote}
Also a section by Debbie.  Also again, not really a theorem and notation should probably be changed (I just changed $S'$ to $B'$ because we use $S'$ for soemthing else). Table needs to be better labeled.  I have similar work for when the plug 1 is allowed that I ran out of time before I put up. 
\end{anote}

\begin{anote}
Use some letter other than $S$ below since that's now used for solution counts.
\end{anote}
\begin{theorem}
Let $B'(n,k)$ represent the number of set partitions with no singletons, i.e. the number of solutions to the plug problem with $k$ plugs, where the plug $1$ is not allowed. Then $B'(n,k) = kB'(n-1,k) + (n-1)B'(n-2,k-1)$, with $B'(2,1)=1$ and $B'(n,k) =0$ when $k > \frac{n}{2}$
\end{theorem}

\begin{proof}
Consider adding position $n-1$ in a strip of length $n$ (where position numbering starts from $0$) to form a solution with $k$ plugs. We can add a prong at position $n-1$ to one of the existing plugs in a solution for a strip of length $n-1$ with $k$ plugs in $kB'(n-1,k)$, which counts all possibilities where position $n-1$ is part of a plug with 3 or more prongs. If position $n-1$ is part of a plug with 2 prongs, then we have $n-1$ choices for the other end of the plug, and we have a total of  $(n-1)B'(n-2,k-1)$ solutions. 
\end{proof}

The solutions form a triangle, with $n$ on the vertical axis and $k$ on the horizontal:

\begin{center}
\begin{tabular}{ |c|c|c|c|c| } 
 \hline
 1 & 2 & 3  & 4 & 5\\ 
 \hline
1 & 0 & 0  & 0 & 0\\ 
2 & 1 & 0  & 0 & 0\\ 
3 & 1 & 0  & 0 & 0\\ 
4 & 1 & 3  & 0 & 0\\ 
5 & 1 & 10  & 0 & 0\\ 
6 & 1 & 25  & 15 & 0\\ 
7 & 1 & 56  & 105 & 0\\ 

 \hline
\end{tabular}
\end{center}
\begin{theorem}
Let $B'(n)$ be the sum of the numbers in row $n$ in the table above, i.e. the total number of strips of length $n$ with singletons disallowed. The first few values of $B'(n)$ are $0,1,1,4,11,21,162$. Let $B(n)$ be the $nth$ Bell Number, which counts the total number of strips of length $n$. Then
\begin{equation*}
   B'(n+1)=B(n)-B'(n), 
\end{equation*}
that is, the number of strips of length $n+1$ that \underline{do not} include a singleton plug is equal to the number strips of length $n$ that \underline{do} include a singleton plug. 

\end{theorem}

\begin{proof}
Let $\mathscr{B}'_n$ and $\mathscr{B}''_n$ be respectively, the sets of all strips of length $n$ without singletons and strips of length $n$ with singletons. We define a map from $\mathscr{B}''_n$ to $\mathscr{B}'_{n+1}$ by considering a strip of length $n$ with at least one singleton, and then  creating a new plug by joining all the singletons with prong $n+1$. This plug is not a singleton and together with the non-singleton plugs from the original, creates a new strip in $\mathscr{B}'_{n+1}$. The map is easily reversible, by taking a strip of length $n+1$ with no singletons, dropping prong $n+1$ and making everything it's connected to into singletons. Thus we've found a bijection. 
\end{proof}

\begin{anote}
Language above needs cleaning up, and not sure about notation either. 
\end{anote}

\section{Puzzles with just one plug type}

\begin{anote}
When we do enough of these we may see metapatterns.
\end{anote}
\begin{itemize}
    \item $\plug{11\ldots 1}$
    This puzzle has just one prime, the plug itself, of length $k$. 
    \item $\plug{100\ldots 1}$ These are the classic plugs. 
    \begin{conjecture}
    Using just the classic plug of length $k$, the only prime is the  one of length $2k-2$ constructed from $k-1$ copies in the obvious way.
    \end{conjecture}
    
    \item $\plug{1011}$
    This puzzle has no solutions.
    \item $(a,a, b, a, a)$ tiles an infinite strip, but has no primes. Note that this \emph{block} notation (which maybe we should change) represents a block of ones of length $a$ followed by a block of zeroes of length $a$, etc. An example is $\plug{1100111110011}$.
    \item $(k,mk,k,mk,...,mk,k)$ The plug formed with $n$ blocks of 1's of length $k$ each and $n-1$ blocks of zeroes of length $mk$ each, where $k,m \in {\mathbb N}$, then $m+1$ of these plugs form a prime of length $(m+1)nk$.
    
    \begin{anote}
    (DB)This is a generalization of Adam's result for the classic plug. The notation probably needs some clean up. I have been using the subset of positions notation to represent plugs. Then if $P$ is the set of prong positions with first prong at 0, we can use $P_k$ to represent the plug that adds $k$ to every number in $P$ ($k$ can be negative, and more generally we don't have to use $P_0$ in our puzzle solution -- I think this will be important if we want to use this notation for puzzles using more than one type of plug) and we are looking for a pairwise disjoint set of these translations that covers an interval. In the case of the above example, we have $P_0 \cup P_k \cup P_{2k} \ldots \cup P_{mk}$ covers the interval $[0,(m+1)nk]$ -- someone please check as there are a lot of variables here. I am also thinking about defining a scaling operation on plugs that multiplies the lengths of the blocks of 0's and 1's by the same thing. If we looked at the strips as continuous instead of discrete, then this change really does just change the unit, it doesn't change how the plugs fit together. Maybe there's another operation that would just duplicate aspects of the plug.... thinking about simpler ways to think about what I wrote here....  these are related to the classic plugs if you can just multiply the block sizes by the same thing and then also sort of repeat the same thing.... 
    \end{anote}
    \begin{conjecture}
    The plugs of the form $(k,mk,k,mk,...,mk,k)$ are the only plugs that form primes with only one type of plug. 
    \end{conjecture}
    

        \end{itemize}
\begin{question}
Are there any single plug puzzles with more than one prime?
\end{question}

\begin{question}
Can we characterize the single plug puzzles with no solutions?
\end{question}


\section{Puzzles with two plug types}


\section{Puzzles using 2 Prong Plugs}

\begin{anote}
There seem to be two attempts at this count. Perhaps one from Adam (moved here) or perhaps both from Deb. I'm not sure what is being counted. We should look at this with our codified definitions.

First attempt:
\end{anote}

   Using only classic plugs of length $2^k$ where $k>0$:
        \begin{center}
        \begin{tabular}{ |c|c|c|c|c| } 
         \hline
         length & $S'$ & $S$ \\ 
         \hline
        2 & 1 & 1\\
        4 & 1 & 2\\
        6 & 1 & 4\\
        8 & 3 & 11\\
        10 & 12 & 33\\
        12 & 14 & 86\\
        14 & -- & --\\
         \hline
        \end{tabular}
        \end{center}
        
\begin{anote}
Deb's work:
\end{anote}

In this section we study the two prong plug puzzle
\begin{equation*}
    \mathbb{T} = \puzzle(\{T_k | k = 1, 2, \ldots\})
\end{equation*}
This is a variant of the stackexchange puzzle, using $T_0$ instead of the single prong plug \plug(1), with no restrictions on the number of each kind of plug.

\begin{anote}
What is below doesn't really warrant a theorem... maybe a proposition or a lemma or just within the text, not sure. Also the $t(n)$ notation is temporary pending my comment in the Specifying Variations section. Also should we use length $n$ for the strip for consistency, which would make all odd lengths 0, or use $2n$ as I did below?
\end{anote}

\begin{theorem}
For the puzzle $\mathbb{T}$ there are no solutions of odd length. For each $n$,
\begin{equation*}
    S(2n)= (2n-1)!! = (2n-1) \cdot (2n-3) \cdots 3 \cdot 1.
\end{equation*}
\end{theorem}

\begin{proof}
Solutions of length $2n$ correspond to   partitions of the set of $2n$ slots into subsets of size two. There are $2n-1$ ways to choose a partner for the plug whose first prong occupies the first slot , $2n-3$ ways to choose a partner for the plug whose first prong occupies the next empty slot, and so on. 
\end{proof}


\begin{comment}
This is wrong but keeping for the LaTex when I rewrite: Note that the exponential generating function for $t(n) = \sum _{i} \dfrac{t(i)}{i!}x^i= e^{\frac{x^2}{2}}$.


\begin{anote}
Probably in a paper, the above would just be referenced, but here's more on this for us. Look at the expansion $e^{\frac{x^2}{2}}= 1+ \dfrac{(\dfrac{x^2}{2})^1}{1!}+\dfrac{(\dfrac{x^2}{2})^2}{2!}+\dfrac{(\dfrac{x^2}{2})^3}{3!}+\cdots $  OK, I need to think about how to explain this better.... want to write up other parts now, but will go back to it later. Also I think I'm using the wrong latex commands for some of these, they don't look so good...
\end{anote} 
\end{comment}


Theorem~\ref{thm:primerecursion} applied to the sequence
\begin{equation*}
    0, 1, 0, 3, 0, 15, 0, 105, 0, 945, 0, 10395, \ldots
\end{equation*} 
shows that the number of length $n$ prime plugs for this puzzle is the sequence
\begin{equation*}
    0, 1, 0, 2, 0, 10,  0, 74, 0, 706, 0, 8162, \ldots 
\end{equation*}

\begin{conjecture}
The nonzero entries in that sequence match \href{https:oeis.org/A000698}{A000698} in the OEIS. The comment there explictly mentions the convolution construction from the Bell numbers 
\end{conjecture}

We obtain the following table with the first few nonzero values of $S(n)$, along with the ratio $\dfrac{S'(n)}{S(n)}$.

\begin{center}
\begin{tabular}{ |c|c|c| } 
 \hline
 $n$ & $S(n) $ & $\dfrac{S'(n)}{S(n)}$ \\ 
 \hline
 1 & 1 & 1 \\ 
 2 & 2 & 0.67 \\ 
 3 & 10 & 0.67 \\ 
 4 & 74 & 0.70 \\ 
 5 & 706 & 0.75 \\ 
6 & 8162 & 0.79 \\ 
 \hline
\end{tabular}
\end{center}


\section{\plug{$11\ldots1$}}

 \begin{anote}
 Some thoughts about the constant plugs. Not sure how deep or interesting they are or might be.
 \end{anote}
 
 Let  
 \begin{equation*}
     R_k = \plug{$11\ldots1$}\ \   (k \  \plug{1}\text{'s})
 \end{equation*}
 
 The $S(n)$ solutions to the puzzle $\puzzle{R} = \puzzle{E}(\text{all }  R_k)$ correspond to the \emph{compositions} of $n$: ways to write $n$ as an ordered sum of nonnegative integers. Thus $S(n) = 2^{n-1}$.
 
 The only prime solutions are the plugs themselves, so $S'(n) = 1$ for all $n$. Convolution confirms that.
 
 The puzzles
 \begin{equation*}
     \puzzle{R}(s,t) = \puzzle{E}(\{R_s, R_t\})
 \end{equation*}
 are a little more interesting. We can assume $s$ and $t$ are relatively prime. Then $S(n)$ is the number of nonnegative integral solutions to the diophantine equation
 \begin{equation*}
     sx + ty = n .
 \end{equation*}
 when the order of the $s+t$ summands is taken into account.
 For example, when $s=3, t=5$ it's the sequence
  \begin{equation*}
    0, 0, 1, 0, 1, 1, 0, 2, 1, 1, 3, 1, 3, 4, 2, 6, 5, 5, 10, 7, 11, 15, \ldots
 \end{equation*}
 with the a two term recursion
  \begin{equation*}
     S(n) = S(n-3) + S(n-5).
 \end{equation*}
 
 In general, the number of solutions to $\puzzle{R}(s,t)$ grows quadratically. The number of solutions $S(n)$ for a subset of $m$ of the plugs $R_k$ will grow at rate $n^m$.  
\section{Asymptotics}

\begin{conjecture}
Let $\mathcal{P}$ be a set of plugs.
The number of solutions $S(n)$ to $\puzzle{E}(\mathcal{P})$ grows as $n^{|\mathcal{P}|}$  if $\mathcal{P}$ is finite, exponentially otherwise.
\end{conjecture}

\section{Infinite strips}

\begin{anote}
Placeholder for work on infinite prime strips.
\end{anote}

With the plugs $A = $\plug{10101}$ and $B = \plug{1011101} you can construct many doubly infinite filled strips since each interlocks with itself and with the other at either end. 

The sequence of $A$'s and $B$'s can be made aperiodic in many ways. For example, start with the singly infinite
\begin{equation}
    ABAABAAABAAAAB\ldots
\end{equation}
then link it to its reverse.

In fact, starting with the \href{https://en.wikipedia.org/wiki/Thue-Morse_sequence}{Thue Morse sequence}
\footnote{ Recursively, start with $A$. For each sequence of length a power of 2, make a new version swapping $A$ and $B$ and append.}
you can build such an aperiodic strip without ever using three of a kind in a row:  

\begin{equation*}
  A B  BA BAAB BAABABBA \ldots  
\end{equation*}

\begin{conjecture}
This sequence is clearly aperiodic in $A$ and $B$. Check that it is also aperiodic in bits.
\end{conjecture}
\section{Convolution}\ref{sec:convolution}

The calculation in Theorem\ref{thm:primerecursion} is (potentially) interesting for its own sake, not just for our applications to plug problems. The functions t2p() and p2t() that convert between total counts and prime counts each invert the other, so each is bijective when applied to arbitrary sequences of integers (or sequences of real numbers, or sequences of elements from any ring).

p2t always increases entries. It's a kind of integration, counting products given prime factors. Its inverse is a kind of differentiation. 

If you play with interesting sequences you can find interesting results/coincidences. For example

\begin{verbatim}
> python3.6 convolve.py   1 0 1 0 1 0 1 0 1
input [1, 0, 1, 0, 1, 0, 1, 0, 1]
p2t: [1, 1, 2, 3, 5, 8, 13, 21, 34]
t2p [1, -1, 2, -3, 5, -8, 13, -21, 34]
\end{verbatim}

When you convolve some counting sequence the result may count something related. There are probably theorems lurking here.

The Catalan numbers (1,1,2,5,14,42,132\ldots) convolve almost to themselves under p2t --- only the leading $1$ is dropped. An interesting application of the Catalan numbers is that the $n$th Catalan number $C_n$ for $n\geq 0$ counts the number of valid ways to arrange $n$ open parentheses and $n$ closed parentheses.

Suppose we introduce an additional count $C_n'$ for the number of prime arrangements of $n$ open parentheses and $n$ closed parentheses (i.e. arrangements in which the initial open parenthesis and final closed parenthesis are linked). It follows that $C_n'$ is just the number of ways to arrange the parentheses on the interior, of which there are n-1 of each. So, $C_{n+1}'=C_n$.

It's easy to see that $C_1=C'_1=1$ and thus $C'_2=C_1=1.$ Passing the list $[C'_1, C'_2]=[1,1]$ to p2t() in convolve.py outputs $[C_1, C_2]=[1,2]$. This output also tells us $C'_3=C_2=2$ — continuing this process we construct the Catalan numbers (omitting the leading 1) through convolution, as shown in full below:

\begin{verbatim}
python3.6 convolve.py  1 1 2 5 14 42 132 429 1430
input [1, 1, 2, 5, 14, 42, 132, 429, 1430]
p2t: [1, 2, 5, 14, 42, 132, 429, 1430, 4862]
t2p [1, 0, 1, 2, 6, 18, 57, 186, 622]
\end{verbatim}
 Under t2p they convolve to 
\href{https://oeis.org/A000957}{OEIS A000957}:
\begin{verbatim}
	Fine's sequence (or Fine numbers):
	number of relations of valence >= 1 on an n-set;
	also number of ordered rooted trees with n edges 
	having root of even degree. 
	
\end{verbatim}
p2t and t2p each essentially permutes the set of sequences. Is there any global structure that's revealed?

\begin{question}

p2t  and  t2p  each  essentially  permutes  the  set  of  sequences.   Is  there  any global structure that’s revealed?
\end{question}
\begin{question}

\end{question}
Is there a continuous form of this kind of self convolution? Something like
\begin{equation*}
    g(x) = \int_0^x f(t)f(x-t)dt
\end{equation*}

\begin{theorem}
Let $S(n)$ and $S'(n)$ be defined as in Theorem~\ref{thm:primerecursion}, with $S(0)=1$ and $S'(0)=0$. Let 
\begin{equation*}
    S(x)=S(0)+S(1)x+S(2)x^2 + \ldots S(n)x^n + \ldots
\end{equation*}
and 
\begin{equation*}
  S'(x)=S'(0)+S'(1)x+S'(2)x^2 + \ldots S'(n)x^n + \ldots  
\end{equation*}
be the ordinary generating functions for $S(n)$ and $S'(n)$ respectively. Then 
\begin{equation*}
    S(x) = \dfrac{1}{1-S'(x)}
\end{equation*}

\begin{anote}
it would be better to use lower case letters for the sequences and upper case for the generating function -- less confusing and more standard. 
\end{anote}
\end{theorem}
\begin{proof}
We rewrite the equation in Theorem~\ref{thm:primerecursion} as 
\begin{equation*}
    S(x)-1=S'(x)S(x)
\end{equation*}
and solve for $S(x)$. See
\end{proof}

\begin{thebibliography}{100}

\bibitem{originalproblem}
Number of ways to arrange pairs of integers with distance constraint
\href{https://math.stackexchange.com/questions/4124452/number-of-ways-to-arrange-pairs-of-integers-with-distance-constraint}{math.stackexchange.com/questions/4124452/number-of-ways-to-arrange-pairs-of-integers-with-distance-constraint}

\end{thebibliography}
%\newpage
%\LaTeX source.
%\verbatiminput{\jobname}
\end{document}