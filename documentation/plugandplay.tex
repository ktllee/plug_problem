\documentclass[10pt]{article}
%\usepackage[textheight=10in, textwidth=7in]{geometry}

\usepackage{amsmath}
\usepackage{amsthm}
\usepackage{amsfonts} % to get \mathbb letters
\usepackage{xcolor}
\usepackage{array}
\usepackage{graphicx}
\usepackage{tabularx}% http://ctan.org/pkg/tabularx
\usepackage{booktabs}% http://ctan.org/pkg/booktabs
\usepackage{hyperref}
\usepackage{verbatim}
\usepackage{comment}

\newtheorem{theorem}{Theorem}
\newtheorem{definition}[theorem]{Definition}
\newtheorem{lemma}[theorem]{Lemma}
\newtheorem{conjecture}[theorem]{Conjecture}
\newtheorem{corollary}[theorem]{Corollary}
\newtheorem{remark}[theorem]{Remark}
\newtheorem{question}{Question}
\newtheorem{example}[theorem]{Example}
\numberwithin{equation}{section}

\DeclareMathOperator{\cp}{cp} % count plugs
\DeclareMathOperator{\cpp}{cpp} % count prime plugs
\DeclareMathOperator{\cs}{cs} % count strips
\DeclareMathOperator{\csp}{csp} % count prime strips

\newenvironment{anote}
               {{\textcolor{blue}{Note:}}
                 \itshape
               }
               {}
                 
\newcommand{\plug}[1]{%
\mbox{{\textbf\texttt #1}}
}

\newcommand{\puzzle}{%
   \ensuremath{
   \mathbb{E}
   }
}

\title{Plug and Play}
\author{
Ethan D. Bolker \\
Debra K. Borkovitz\\
  Katelyn Lee\\
  Adam Salachi
}
\date{\today}

\begin{document}

\maketitle

\section{Introduction}

\begin{anote}
Describe the source of the problem and the group makeup
\end{anote}



\section{Plugs, Strips and Puzzles}

Our mathematical model for a plug is a bit string, for example
\plug{1011}. Our current convention is that there are no leading or
trailing \plug{0} bits. (We may want to relax that convention at some
time.)

The \emph{length} of a plug is the length of the string; this one has
length $4$. The \emph{number of prongs} is the number of \plug{1} bits
--- $3$ in this example.

Each plug has a \emph{plugnumber} when we
interpret its bit string as a binary integer. So \plug{1011} is
number $1 + 4 + 8 = 13$.%
\footnote{This convention reads increasing bit significance from left
  to right, so the first bit is the units bit. We can change our minds
  and set the plugnumber for \plug{1011} to $1+2+8 = 11$ if we wish,
  but let's decide soon and stick to our decision.}

Since plugs begin and end with \plug{1} bits,
the plugs of length $n$ have plugnumbers the odd integers between
$2^{n-1} +1$ and $2^n -1$.
\footnote{This equivalence will need revision if we allow leading and
  trailing \plug{0}s in plug bitstrings.}

We will want ways to talk about plugs other than specifying their bit
strings or plugnumbers. In any context we can give a plug any name we
like, or have variables whose values can be plugs.

The Stackexchange question that triggered our project asked about
the $2$ prong plugs. We'll call those the \emph{classic plugs} and give
them their own names: $T_k$ for the two prong plug of length $k$, It
has $k-2$ \plug{0} bits between its $2$ end prongs. $T_k$ has plugnumber $2^{k-1} + 1$.
 The stackexchange question uses the single prong plug \plug{1}{} instead of the mathematically more natural length $2$ two prong plug $T_2$ with no gap between the prongs.
 
 A \emph{power strip} (or just a \emph{strip}) models a place to plug
 in plugs. Think of it as a finite sequence of slots some of
 which are occupied by the prongs of plugs. So a power strip
is a pair consisting of an array of slots and a set of (plug, offset)
pairs such that all the plugs can be inserted simultaneously at
 the specified offsets. It is a \emph{solved} if
 every slot in the strip is filled.

A \emph{plug puzzle}, or, for us, just a puzzle, specifies the types amd numbers of plugs you are allowed to use to fill a strip --- something like
\begin{equation*}
    \puzzle(\text{list of allowable plug types},\text{restrictions on numbers of plugs of each type})
\end{equation*}

A \emph{solution} of length $n$ is a strip with $n$ slots filled with plugs that are consistent with the restrictions in the puzzle specification.


Given a puzzle we try to understand the number and shape of solutions. often as
  a function of the size of the strip.

\begin{anote}
Remember to say somewhere (perhaps not here) that plugs and solutions have a left to right direction. You can't turn them around. Perhaps add an electric system not quite analogy.
\end{anote}

The stackexchange post posed a sequence of puzzles:
\begin{equation*}
\mathbb{T}_k = \puzzle( (\text{the one prong plug}, T_3, T_4, \ldots. T_k), \text{exactly 1 of each } ).
\end{equation*}
The only possible solutions are strips of length $2k+1$. The question asks for an efficient algorithm to count them.

 \section{Anything goes}
 
That puzzle spurred our investigations,bput it is much too hard to start out with. In particular, we discovered that restricting the number of plugs of each type is a stumbling block. So for a while we will study just puzzles that specify the allowed plug types, with no limit on the number of each that can appear in a solution.

In the \emph{anything goes}  puzzle $\mathbb{A}$  you may use any plugs as oftem as you wish:
\begin{equation*}
  \mathbb{A}  = \puzzle(\text{all plug types } )
\end{equation*}
Counting solutions begins to put us in touch with some classical notions
in combinatorics.

\begin{theorem} Length $n$ solutions for the plug
  puzzle that allows arbitrary many instances of any plug
correspond to the partitions of the   $n$-element set $[n] = \{1,2,
\ldots, n \}$. 
\end{theorem}

\begin{anote}
Example here --- wait until the good \LaTeX{} plug representation
  is done.
\end{anote}

The \href{https://en.wikipedia.org/wiki/Bell_number}{Bell numbers}
$\{1, 2, 3, 5, 15, 52, \ldots$  count these.

\section{Factoring}

There is more information in a a solved  strip than the shape of
the partition that defines it. For example, two copies of \plug{101}
solve a strip of length $4$. So do two copies of \plug{11}. Both solutions correspond to a partition the four element set of slots
into two sets of two slots each, but their geometry is different.

In a  power strip of length $n$ there are $n-1$ \emph{gaps} between
slots. The \emph{thickness} or \emph{thickness array} is a the
sequence of integers that counts the number of plugs that cover each
of the gaps. A solution (that is, a filled strip) is \emph{prime} if
its thickness is never $0$. Any solution is a concatenation of prime
filled strips: the \emph{factors}.

When a puzzle specifies no restrictions on the number of each kind of plug then factors and products (concatenations) of solutions are themselves solutions.  When that  happens we can relate the number of prime solutions and the number of solutions.

Let $\cs(n)$ be the number of solved strips of length $n$ and
$\csp(n)$ the number of prime solutions of length $n$. 
Let $\cp(n)$ be the number of solved strips that use $n$ plugs and
$\cpp(n)$ the number of prime solutions that use $n$ plugs.

Some of these counts may be $0$. Consider the
puzzle that uses just the classic  plug \plug{11}.
Then $cs(n) = 1$ if $n$ is even, $0$ if $n$ is odd. $csp(2) =1$ and
$0$  otherwise.

We did a brute force calculation for the number of prime solutions to the anything goes puzzle, and found $\{1,1,2,6,22, \ldots\}$. That sequence seems to be
\href{https://oeis.org/A074664}{A074664} in
\href{https://oeis.org}{The On-Line Encyclopedia of Integer
  Sequences}. The comments there about permutations and partitions
strongly support

\begin{conjecture}
The prime solutions to problem $\mathbb{A}$ are characterized by the
equivalent conditions defining the OEIS sequence A074664.
\end{conjecture}

This easy theorem might help prove the conjecture:

\begin{theorem}\label{thm:primerecursion}
Suppose that every factor of a solved strip solves its smaller strip and any concatenation of solved strips is a solution.
 Then
$$
\cs(n) = \csp(1)\cs(n-1) +  \csp(2)\cs(n-2) + \cdots + \csp(n-2)\cs(2) + \csp(n)
$$
and the same recursion hold for the counts $\cp)$ and $\cpp)$.
\end{theorem}
\begin{proof}
Each solved strip begins with a prime solution of length $k \le n$ which can be
chosen $\csp(k)$ ways and ends with one of the $\cs(n-k)$ solved strips of
length $n-k$. 
Replace "length" by "nimber of plugs" to prove the second assertion.
\end{proof}

\begin{anote}
So far we've used thickness only to find factors. But there is
probably more information to exploit. What's the maximum thickness?
What patterns are possible in the thickness vector?
\end{anote}

\section {Disallowing the Plug $1$}

\begin{anote}
Also a section by Debbie.  Also again, not really a theorem and notation should probably be changed. Table needs to be better labeled.  I have similar work for when the plug 1 is allowed that I ran out of time before I put up. 
\end{anote}

\begin{theorem}
Let $S'(n,k)$ represent the number of set partitions with no singletons, i.e. the number of solutions to the plug problem with $k$ plugs, where the plug $1$ is not allowed. Then $S'(n,k) = kS'(n-1,k) + (n-1)S'(n-2,k-1)$, with $S'(2,1)=1$ and $S'(n,k) =0$ when $k > \frac{n}{2}$
\end{theorem}

\begin{proof}
Consider adding position $n-1$ in a strip of length $n$ (where position numbering starts from $0$) to form a solution with $k$ plugs. We can add a prong at position $n-1$ to one of the existing plugs in a solution for a strip of length $n-1$ with $k$ plugs in $kS'(n-1,k)$, which counts all possibilities where position $n-1$ is part of a plug with 3 or more prongs. If position $n-1$ is part of a plug with 2 prongs, then we have $n-1$ choices for the other end of the plug, and we have a total of  $(n-1)S'(n-2,k-1)$ solutions. 
\end{proof}

The solutions form a triangle, with $n$ on the vertical axis and $k$ on the horizontal:

\begin{center}
\begin{tabular}{ |c|c|c|c|c| } 
 \hline
 1 & 2 & 3  & 4 & 5\\ 
 \hline
1 & 0 & 0  & 0 & 0\\ 
2 & 1 & 0  & 0 & 0\\ 
3 & 1 & 0  & 0 & 0\\ 
4 & 1 & 3  & 0 & 0\\ 
5 & 1 & 10  & 0 & 0\\ 
6 & 1 & 25  & 15 & 0\\ 
7 & 1 & 56  & 105 & 0\\ 

 \hline
\end{tabular}
\end{center}


\section{Puzzles with just one plug type}

\begin{anote}
When we do enough of these we may see metapatterns.
\end{anote}
\begin{itemize}
    \item $\plug{11\ldots 1}$
    This puzzle has just one prime, the plug itself, of length $k$. 
    \item $\plug{100\ldots 1}$ These are the classic plugs. 
    \begin{conjecture}
    Using just the classic plug of length $k$, the only prime is the  one of length $2k-2$ constructed from $k-2$ copies in the obvious way.
    \end{conjecture}
    
    \item $\plug{1011}$
    This puzzle has no solutions.
    

        \end{itemize}
\begin{question}
Are there any single plug puzzles with more than one prime?
\end{question}

\begin{question}
Can we characterize the single plug puzzles with no solutions?
\end{question}
\section{Puzzles with two plug types}


\section{Puzzles using 2 Prong Plugs}

\begin{anote}
There seem to be two attempts at this count. Perhaps one from Adam (moved here) or perhaps both from Deb. I'm not sure what is being counted. We should look at this with our codified definitions.

First attempt:
\end{anote}

   Using only classic plugs of length $2^k$ where $k>0$:
        \begin{center}
        \begin{tabular}{ |c|c|c|c|c| } 
         \hline
         length & $\csp$ & $\cs$ \\ 
         \hline
        2 & 1 & 1\\
        4 & 1 & 2\\
        6 & 1 & 4\\
        8 & 3 & 11\\
        10 & 12 & 33\\
        12 & 14 & 86\\
        14 & -- & --\\
         \hline
        \end{tabular}
        \end{center}
        
\begin{anote}
Deb's work:
\end{anote}

In this section we count strips that can be made eith just the two prong plugs:
\begin{equation*}
    \mathbb{T} = \puzzle(\{T_k | k = 1, 2, \ldots\})
\end{equation*}
This is the stackexchange puzzle, using $T_0$ instead of the single prong plug \plug(1), with no restrictions on the number of each kind of plug.

\begin{anote}
What is below doesn't really warrant a theorem... maybe a proposition or a lemma or just within the text, not sure. Also the $t(n)$ notation is temporary pending my comment in the Specifying Variations section. Also should we use length $n$ for the strip for consistency, which would make all odd lengths 0, or use $2n$ as I did below?
\end{anote}

\begin{theorem}
For the puzzle $\mathbb{T}$ there are no solutions of odd length. For each $n$,
\begin{equation*}
    \cs(2n)=(2n-1) \cdot (2n-3) \cdots 3 \cdot 1.
\end{equation*}
\end{theorem}

\begin{proof}
Solutions of length $2n$ correspond to   partitions of the set of $2n$ slots into subsets of size two. There are $2n-1$ ways to choose a partner for the plug whose first prong occupies the first slot , $2n-3$ ways to choose a partner for the plug whose first prong occupies the next empty slot, and so on. 
\end{proof}


\begin{comment}
This is wrong but keeping for the LaTex when I rewrite: Note that the exponential generating function for $t(n) = \sum _{i} \dfrac{t(i)}{i!}x^i= e^{\frac{x^2}{2}}$.

\begin{anote}
Ethan. I rewrote this section up to this point. I think the next argument is really using the theorem that relates the total count to the count of primes.

When Deb or someone else reworks the rest of this section the they should work out counts of number of plugs (always $n$0 and number of prime plugs ?? 
\end{anote}

\begin{anote}
Probably in a paper, the above would just be referenced, but here's more on this for us. Look at the expansion $e^{\frac{x^2}{2}}= 1+ \dfrac{(\dfrac{x^2}{2})^1}{1!}+\dfrac{(\dfrac{x^2}{2})^2}{2!}+\dfrac{(\dfrac{x^2}{2})^3}{3!}+\cdots $  OK, I need to think about how to explain this better.... want to write up other parts now, but will go back to it later. Also I think I'm using the wrong latex commands for some of these, they don't look so good...
\end{anote} 
\end{comment}


We directly compute several values of $\csp(n)$, the number of prime strips of length $2n$ , using the "brute force" equation that subtracts composite plugs corresponding to all compositions of $n$ into $k$ parts, where $ k \leq n$, 

\begin{equation*}
    pt(n)=t(n)-\sum _{i_1+i_2 + \ldots i_k=n}pt(i_1)pt(i_2) \cdots pt(i_k)
\end{equation*}

For example, we note that $pt(1) =1$ and compute $\csp(2) = \cs(2) -pt(1)pt(1) = 3 - 1 \cdot 1 = 2$. Also we have $pt(3) = t(3) - pt(1)pt(2) - pt(2)pt(1) - pt(1)pt(1)pt(1) = 15 - 2 - 2- 1 = 10$.

\begin{anote}
Ethan: is this essentially Theorem~\ref{thm:primerecursion}?
\end{anote}
We obtain the following table with the first few values of $pt(n)$, along with the ratio $\dfrac{pt(n)}{t(n)}$.

\begin{center}
\begin{tabular}{ |c|c|c| } 
 \hline
 $n$ & $pt(n) $ & $\dfrac{pt(n)}{t(n)}$ \\ 
 \hline
 1 & 1 & 1 \\ 
 2 & 2 & 0.67 \\ 
 3 & 10 & 0.67 \\ 
 4 & 74 & 0.70 \\ 
 5 & 706 & 0.75 \\ 
6 & 8162 & 0.79 \\ 
 \hline
\end{tabular}
\end{center}

\section{Asymptotics}

\begin{anote}
Placeholder for work on asymptotic estimates or bounds for puzzles where we can make some progress but can't find  good closed form or known desriptions of the counting functions.
\end{anote}

\section{Infinite strips}

\begin{anote}
Placeholder for work on infinite prime strips.
\end{anote}
%\newpage
%\LaTeX source.
%\verbatiminput{\jobname}
\end{document}