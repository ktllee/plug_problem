\documentclass[10pt]{article}
%\usepackage[textheight=10in, textwidth=7in]{geometry}

\usepackage{amsmath}
\usepackage{amsthm}
\usepackage{amsfonts} % to get \mathbb letters
\usepackage{xcolor}
\usepackage{array}
\usepackage{graphicx}
\usepackage{tabularx}% http://ctan.org/pkg/tabularx
\usepackage{booktabs}% http://ctan.org/pkg/booktabs
\usepackage{hyperref}
\usepackage{verbatim}

\newtheorem{theorem}{Theorem}
\newtheorem{definition}[theorem]{Definition}
\newtheorem{lemma}[theorem]{Lemma}
\newtheorem{conjecture}[theorem]{Conjecture}
\newtheorem{corollary}[theorem]{Corollary}
\newtheorem{remark}[theorem]{Remark}
\newtheorem{question}{Question}
\newtheorem{example}[theorem]{Example}
\numberwithin{equation}{section}


\newcommand{\plug}[1]{%
\mbox{{\textbf\texttt #1}}
}

\title{Plug and Play}
\author{
Ethan D. Bolker \\
Deborah Borkovitz\\
  Katelyn Lee\\u
  Adam Salachi
}
\date{\today}

\begin{document}

\maketitle

\section{Introduction}

\emph{Describe the source of the problem and the group makeup}


\section{Plugs}

Our mathematical model for a plug is a bit string, for example
\plug{1011}. Our current convention is that there are no leading or
trailing \plug{0} bits. (We may want to relax that convention at some
time.)

The \emph{length} of a plug is the length of the string; this one has
length $4$. The \emph{number of prongs} is the number of \plug{1} bits
--- $3$ in this example.

We will want ways to talk about plugs other than specifying their bit
strings or plugnumbers. In any context we can give a plug any name we
like, or have variables whose values can be plugs.

Each plug has a \emph{plugnumber} when we
interpret its bit string as a binary integer. So \plug{1011} is
number $1 + 4 + 8 = 13$.%
\footnote{This convention reads increasing bit significance from left
  to right, so the first bit is the units bit. We can change our minds
  and set the plugnumber for \plug{1011} to $1+2+8 = 11$ if we wish,
  but let's decide soon and stick to our decision.}

Since plugs begin and end with \plug{1} bits,
the plugs of length $n$ have plugnumbers the odd integers between
$2^{n-1} +1$ and $2^n -1$.
\footnote{This equivalence will need revision if we allow leading and
  trailing \plug{0}s in plug bitstrings.}

The Stackexchange question that triggered our project asked about
the $2$ prong plugs. We will give
them their own names: $T_n$ for the two prong plug of length $n$, It
has $n-2$ \plug{0} bits between its $2$ end prongs.
\footnote{Perhaps we want the nontraditional $n_T$, emphasizing the
  $n$ in our name for the two prong plug of that length.}

That question used the the single prong plug number $1$ instead of the
double prong plug $T_2$ with no gap between its prongs.

\section{Plug Puzzles}

A \emph{plug puzzle} is a finite multiset\footnote{A multiset is a
  ``set that allows for repetitions of elements''.}
 of plugs. A \emph{labeled} plug puzzle is a set of labeled
 plugs. The fact that it's a set implies that all the labels are
 distinct. When no two of the plugs have the same shape (bit string)
 the strings or the plug numbers can serve as labels.

 \begin{equation*}
\{\plug{1011}, \plug{1001}, \plug{100} \}
\end{equation*}
 The distinction between ordinary and labeled plug sets is analogous
 to the distinction common in combinatorial counting questions where
 the balls being distributed into boxes are identical or
 distinguishable.

 In Python we would model a labeled plug puzzle as a dictionary, with
 key the label and value the plug.

 For example, consider the labeled puzzle
 \begin{equation*}
   \{ (a, \plug{1011}),
   (b,\plug{1001}),
   (c, \plug{1001}),
   (d, \plug{101}) \}
\end{equation*}

 The \emph{length} of a plug puzzle is the sum of the lengths of its
 plugs. This puzzle has length $9$.
 
 \section{Power strip}

 A \emph{power strip} (or just a \emph{strip} models a place to plug
 in plugs. Think of it as a finite sequence of slots subsets of which
 can  be occupied by the prongs of a plug. 

 A plug can be \emph{inserted} in a strip at an integer offset $m$ if
 the plugs   \plug{1} bits shifted by $m$ correspond to empty slots.
 A pair consisting of a strip and a set of (plug, offset) pairs
 and a strip is \emph{compatible} if all the plugs can be inserted at
 the specified offsets. It is a \emph{solution} if
 every slot in the strip is filled.

 \section{An easy interesting question}

 The stackexchange question asked for a fast algorithm to count the
 number of solutions for the plug puzzle consisting of the first $n$
 two prong puzzles.
 
We think that problem is too  hard to tackle first. We may return to
it.  Here's an easy one to start with - hardly worth calling a
theorem, but it begins to put us in touch with some classical notions
in combinatorics.

\begin{theorem} For each integer $n$, the plug puzzle solutions for
  the strip of length $n$ correspond to the partitions of the
  $n$-element set $[n] = \{1,2, \ldots, n \}$.
\end{theorem}

\section{Thickness}

There is more to the previous theorem than just counting partitions.

\emph{Define thickness, as a sequence indexed by the spaces bewteen
  slots in a strip. That allows us to consider unbreakable solutions
  (thickness always positive) and maximum thickness.}
%\newpage
%\LaTeX source.
%\verbatiminput{\jobname}
\end{document}
